
Goals:

Rely on the C++ standard to provide platform-independence, because the
C++ standard will work on future platforms without requiring
source-code changes. This means:

* Using C++ functionality rather than POSIX or Windows functionality.

* Choosing a specific C++ standard , and then
specifying that standard to the compiler.


* Remove as many \#ifdef preprocess directives as possible.

* Replacing code that we had painstakingly written, debugged and
maintained with code now supported in the C++ standard

We originally chose C++14 but changed to C++17 to get the
std::filesystem support.


Dramatically improve reliability. This means:
* Whereas previously many strings were passed by reference as 'const std::string &',
there are now passed by value 'std::string'. This necessitates a
string copy, but it is not a meaningful impact on performance,
especially when compared with the improved safety against possibly
using an invalidated reference.

* we defined a clear allocation/deallocation policy for all
objects in memory, especially the sbuf objects.

* We enabled sbuf child tracking. Previously this was turned off
because of an underlying memory allocation bug. Once we defined the
clear policy described above, we were able to find the bug!

* Unit tests and code-coverage, rather than simply end-to-end
regression tests.

* Moved more functionality into the sbuf_t structure, such as
computing the hash of an sbuf. We also now cache the hash, so we are
assured that each block of data will only be hashed once.


Code Quality Goals:
* Significantly reduce the size of the bulk_extractor code base.
* Eliminate global variables used to track state. (Global variables
used to implement static tables and data-driven functions were allowed
to persist, although the code that uses these variables now checks to
verify that they have been initialized and throws an exception if they
have not.)
* Removal of return codes that must be checked to detect
errors. Instead, we use the C++ exception mechanism to signal and
catch error conditions.
* Elimination of explicit mutexs when possible, replaced with the C++
atomic template.

Performance Goals:
* Run twice as fast as the original bulk_extractor when compiled with
the same (modern) compiler with the same number of cores.
* Be able to run on a cluster with a high-performance storage system,
such as Amazon EBS or S3, and process with Amazon Lambda, with the
goal of being able to process a terabyte reference disk image in 5
minutes.

Functionality Goals:
* Link with The Sleuth Kit and automatically process non-contiguous
files if the file system is intact. Automatically report features by
both location on the disk and the file in which they are found. We did
this in a backwards compatiable manner by adding an addition field to
the pos0_t record.
(requires an additional field be added to the feature file).

* Removing the dependence on Java for the user interface.



Removed functionality:
- TO the best of our knowledge, no one (other than the original
developer) ever used bulk_extractor's shared library to let the
program's scanner system be called from C++ or from Python, so we did
dropped support for that. (It could be trivially added in the future.)

- The stand-alone bulk_extractor test program that only scans a single
file was dropped as additional code that did not need to be
maintained. Instead, we now have unit tests.

- The ability to load scanners as shared libraries at startup was
kept, although it is not clear that this has ever been used. It was
created so that bulk_extractor users could develop and maintain their
own scanners without telling the developers of them. This might be
useful if \be was being used in a highly restricted environment. If
such use was ever made, the developers were never told! Such unknown
users are advised that they will need to revise their scanners for
version 2.0.


- Moved common code out of the scanners and into the framework. For
example, rather than each scanner having options for setting its carve
mode, the framework understands how to set them for any named
scanner.

(explain carve mode.)

Simplified the API. sbuf previously had a two map_file's - one that
took filename already open and the fd, and ht eother which just took
the filename. Now it just has one; if it is already open, then it is
oepened a secon time.
